\begin{description}
\item[Existence] : Soient $t_0,t_1,\dots,t_n\in\R$ distincts et soient $y_0,y_1,\dots,y_n\in\R$, il existe un et un seul polynôme $p\in\mathcal P_n$ tel que
    \[
        p_n-(t_i)=y_i, \forall i=0,1,\dots,n
    \]
\item[Dans la base canonique] : Dans la base canonique $(1,x,x^2,\dots,x^n)$ il suffit de résoudre le système $p_i(t_i)=y_i$ i.e.
    \[
        Ax=b\textrm{ avec }
        \begin{cases}
            \underline{A_i} = \begin{pmatrix}1 & t_i & t_i^2 & \dots & t_i^n\end{pmatrix}\\
            \underline{b_i} = \begin{pmatrix}y_i\end{pmatrix}\\
            x = \begin{pmatrix}a_0 & a_1 & \dots & a_n\end{pmatrix}^T
        \end{cases}
    \]
\item[Dans la base de Lagrange] : On appelle base de Lagrange la famille
    \[
        \begin{pmatrix}
            \mathcal L_1(t),
            \mathcal L_2(t),
            \dots,
            \mathcal L_n(t)
        \end{pmatrix}
        \textrm{ où }
        \mathcal L_i(t) = 
        \prod_{k=0, k\ne i}^n \dfrac{t-t_k}{t_i-t_k}
        =
        \begin{cases}
            1\textrm{ si }t_i=t\\
            0\textrm{ sinon}
        \end{cases}
    \]
    Le polynôme d'interpolation est alors donné par
    \[
        p_n(t)=\sum_{i=0}^ny_i\mathcal L_i(t)
    \]
\item[Erreur] : On note $e_n(t)=f(t)-p_n(t)$ l'erreur d'interpolation. On note également $\pi_n(t)=(t-t_0)(t-t_1)\dots (t-t_n)$\\
    Alors (en notant $Int(t_0,\dots,t_n)$ le plus petit interval contenant les $t_0,\dots,t_n$)
    \[
        e_n(t) = \dfrac{\pi_n(t)}{(n+1)!}f^{(n+1)}(\xi)\textrm{ avec }\xi\in Int(t_0,\dots,t_n)
    \]
\item[Dans la base de Newton] : Dans la base de Newton $(1,t-t_0,(t-t_0)(t-t_1),\dots,(t-t_0)(t-t_1)\dots(t-t_{n-1}))$, le polynôme d'interpolation est donné par
    \[
        p_n(t) = c_0
            + c_1(t-t_0)
            + \dots +
            c_n(t-t_0)(t-t_1)\dots(t-t_{n-1})
    \]
    Où les $c_k$ sont les différences divisés d'ordre $k$
\item[Différence divisée] : Soit $f$ une fonction dont on connait les valeurs en des points disincts $t_0,t_1,\dots,t_n$. On appelle différence divisée l'expression suivante :
    \[
        \begin{cases}
            f[a]=f(a)\\
            f[a,X,b]=\dfrac{f[a,X]-f[X,b]}{a-b}
        \end{cases}
    \]
\item[Calcul pratique des différences divisées] : Les coéficients $c_k$ sont sur la diagonale en $k$-ième position \\
    \begin{center}
        \begin{tabular}{c|c|c|c|c}
            $k=0$ & $k=1$ & $k=2$ & $\dots$ & $k=n$ \\\hline
            $f[t_0]$ & & & & \\
            $f[t_1]$ & $f[t_0,t_1]$ & & \\
            $f[t_2]$ & $f[t_1,t_2]$ & $f[t_0,t_1, t_2]$ & & \\
            $\vdots$ & $\vdots$ & $\vdots$ & $\ddots$ & \\
            $f[t_n]$ & $f[t_{n-1},t_n]$ & $f[t_{n-2},t_{n-1},t_n]$ & $\hdots$ & $f[t_0,t_1,\dots,t_n]$
        \end{tabular}
    \end{center}
\item[Schéma de Horner] : Pour calculer $p_3(t)=c_0+c_1(t-t_0)+c_2(t-t_0)(t-t_1)+c_3(t-t_0)(t-t_1)(t-t_2)$, on calcul plutôt
    \[
        p_3(t) = c_0 + (t-t_0)[ c_1 + (t-t_1)[ c_2 + (t-t_2)[ c_3 ] ] ]
    \]
\item[Splines cubiques] : Soit $\Delta = (a=t_0,t_1,\dots,t_n=b)$ une subdivision de l'intervalle $[a,b]$. On dit qu'une fonction $g$ est un spline cubique si
    \begin{itemize}
        \item $g\in C^2([a,b])$
        \item $g$ correspond sur chaque intervalle $[t_i, t_{i+1}]$ à un polynôme de degré inférieur à $3$
        \item $g(t_i)=y_i$
    \end{itemize}
\end{description}