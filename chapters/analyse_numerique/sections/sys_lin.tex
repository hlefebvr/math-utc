\subsection{Algorithme de Gauss}
\begin{description}
\item[Élimination de Gauss] : Soit une matrice $A\in\mathcal M_{nn}$ et $b\in\mathcal M_{n1}$, pour résoudre efficacement l'équation $Ax=b$, on cherche à transformer $A$ en une matrice triangulaire grâce à l'algorithme de Gauss.\\
    On note $A^{(0)}=A$\\
    On trouve alors $A^{(k+1)}$ en fonction de $A^{(k)}$ avec :
    \[
        \begin{cases}
            a_{ij}^{(k+1)}=a_{ij}^{(k)}-\dfrac{a_{ik}^{(k)}}{a_{kk}^{(k)}}a_{kj}^{(k)}\\
            b_{ij}^{(k+1)}=b_{ij}^{(k)}-\dfrac{a_{ik}^{(k)}}{a_{kk}^{(k)}}b_{kj}^{(k)}, \textrm{ pour } j=k,k+1,\dots,n
        \end{cases}
    \]
    L'algorithme termine pour $k=n$
\item[Écriture matricielle de Gauss] : On peut écrire les équations précédentes sous la forme matricielle suivante, en notant $\underline{A_i}$ la $i$-ième ligne de $A$
    \[
        \underline{A_i^{(i}}=\underline{A_i^{(i-1}}-\left(
            \dfrac{a_{i,i-1}^{(i-1)}}{a_{a-1,a-1}^{(i-1)}}
        \right)
        \underline{A_{i-1}^{(i-1)}}
    \]
\item[Pivot] : Les coefficients $a_{kk}^{(k)}$ sont appelées les pivots de Gauss
\end{description}
\subsection{Factorisation de matrices}
\begin{description}
\item[Sous-matrice principale] : On appelle sous-matrice principale de $A$ d'ordre $k$ la matrice notée
    \[
        [A]_k=(a_ij)_{1\le i\le k, 1\le j\le k}
    \]
\item[Factorisation $LU$] : Il s'agit de trouver $U\in\mathcal M_{nn}$ triangulaire supérieur et $L\in\mathcal M_{nn}$ triangulaire inférieur telle que $A=LU$, $L$ ayant tous ses termes diagonaux égaux à $1$
    cf. Algorithme de Doolitle
\item[Existence] :
    \[
        A\textrm{ est } LU\textrm{-factorisable}\Leftrightarrow [A]_1,[A]_2,\dots,[A]_n\textrm{ inversible}
    \]
\item[Unicité] : La factorisation $LU$, si elle existe, est unique
\item[Factorisation $PALU$] : Si $A$ est non $LU$-factorisable, on peut permuter les lignes de $A$ pour effectuer la factorisation.\\
    Le système s'écrit alors $PA=LU\Leftrightarrow C=LU$ avec $C=PA$ et $P$ la matrice carrée indiquant les permutations effectuées
\item[Factorisation $LUPAQ$] : Pour les mêmes raisons, et de manière similaire, on peut permuter les colonnes de $A$.\\
    Le système devient alors $PAQ=LU\Leftrightarrow C=LU$ avec $C=PAQ$
\item[Application à la résolution de systèmes linéaires] : Si $A$ est $LU$-factorisable, alors 
    \[
        Ax=b\Leftrightarrow LUx=b\Leftrightarrow
        \begin{cases}
            Ly=b\\
            Ux=y
        \end{cases}
    \]
    (Ce qui est immédiat puisque $L$ et $U$ sont triangulaires)
\item[Factorisation $LDL^T$] : Si $A$ est $LU$-factorisable et symétrique, alors $\exists L,D\in\mathcal M_{nn}$ avec $L$ une matrice triangulaire inférieur à diagonale unité et $D$ une matrice diagonale telle que 
    \[ A=LDL^T \]
\item[Factorisation de Cholesky $(BB^T)$] : Si $A$ est une matrice symétrique définie-positive alors elle admet une factorisation unique $A=BB^T$ avec $B$ une matrice triangulaire inférieur dont les termes diagonaux sont positifs.\\
    cf. Algorithme de Cholesky
\end{description}