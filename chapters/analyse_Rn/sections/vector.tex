\begin{description}
\item[Produit scalaire] : $\vect{u}.\vect{v}=0\Leftrightarrow\vect{u}$ et $\vect{v}$ sont orthogonaux
    \[
        \vect u.\vect v=\sum_{i=1}^nx_iy_i=||x||.||y||\cos\theta
    \]
\item[Produit vectoriel] : $\vect u\land\vect v=\vect 0\Leftrightarrow\vect u$ et $\vect v$ sont colinéaires
    \[
        \begin{pmatrix}
            u_1\\
            u_2\\
            u_3
        \end{pmatrix}
        \land
        \begin{pmatrix}
            v_1\\
            v_2\\
            v_3
        \end{pmatrix}
        =
        \begin{pmatrix}
            u_2v_3-u_3v_2\\
            u_3v_1-u_1v_3\\
            u_1v_2-u_2v_1
        \end{pmatrix}
    \]
    $\vect u\land\vect v=-\vect v\land\vect u$
\item[Produit mixte] : $(\vect u,\vect v,\vect w)=0\Leftrightarrow\vect u,\vect v$ et $\vect w$ sont coplanaires
    \[
        (\vect u,\vect v,\vect w)=(\vect u\land\vect v).\vect w=\textrm{ volume du parallélépipède formé par }\vect u,\vect v\textrm{ et }\vect{w}
    \]
\item[Coordonées cylindriques] :
    \[
        \begin{cases}
            x=r\cos\theta\\
            y=r\sin\theta\\
            z=z
        \end{cases},
        \theta\in[0,2\pi[
    \]
\item[Coordonées sphériques] :
    \[
        \begin{cases}
            x=\rho\cos\phi\cos\theta\\
            y=\rho\cos\phi\sin\theta\\
            z=\rho\sin\phi
        \end{cases},
        \theta\in[0,2\pi[,
        \phi\in\left[-\frac{\pi}{2},\frac{\pi}{2}\right[
    \]
\item[Gradient] :
    \[
        \vect\nabla f=
        \begin{pmatrix}
            \dpartial{f}{x}\\
            \dpartial{f}{y}\\
            \dpartial{f}{z}\\
        \end{pmatrix}
        \textrm{ ; }
        \vect\nabla(fg)
        =f\vect\nabla g+g\vect\nabla f
    \]
\item[Rotationnel] :
    \[
        \rot\vect V=
        \begin{pmatrix}
            \dpartial{}{x}\\
            \dpartial{}{y}\\
            \dpartial{}{z}\\
        \end{pmatrix}
        \land
        \begin{pmatrix}
            P(x,y,z)\\
            Q(x,y,z)\\
            R(x,y,z)
        \end{pmatrix}
        \textrm{ ; }
        \rot f\vect V=f\rot\vect V+\grad f\land\vect V
    \]
\item[Divergence] :
    \[
        \divg f=\dpartial{f}{x}+\dpartial{f}{y}+\dpartial{f}{z}
        \textrm{ ; }
        \begin{cases}
            \divg f\vect V=f\divg\vect V+\grad f.\vect V\\
            \divg\vect V_1\land\vect V_2=\vect V_2\rot\vect V_1-\vect V_1\rot\vect V_2
        \end{cases}
    \]
\item[Laplacien] :
    \[
        \Delta f=\divg\grad f=
        \frac{\partial^2 f}{\partial x^2}+
        \frac{\partial^2 f}{\partial y^2}+
        \frac{\partial^2 f}{\partial z^2}
    \]
\item[Propositions] :
    \begin{align*}
        &f\in C^2\Rightarrow\rot\grad f=0\\
        &\vect V=(P,Q,R)^T\textrm{ avec }P,Q,R\in C^1, \rot\vect V=0\Rightarrow\exists f\tq\grad f=\vect V\\
        &\vect V=(P,Q,R)^T\textrm{ avec }P,Q,R\in C^2, \divg\rot\vect V=0\\
        &\vect V=(P,Q,R)^T\textrm{ avec }P,Q,R\in C^1, \divg\vect V=0\Rightarrow\exists\vect A\tq\rot\vect A=\vect V
    \end{align*}
\end{description}